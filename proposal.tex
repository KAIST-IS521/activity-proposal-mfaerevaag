\documentclass[a4paper, 11pt]{article}

% \usepackage{kotex} % Comment this out if you are not using Hangul
\usepackage{fullpage}
\usepackage{hyperref}
\usepackage{amsthm}
\usepackage[numbers,sort&compress]{natbib}

\theoremstyle{definition}
\newtheorem{exercise}{Exercise}

\begin{document}
%%% Header starts
\noindent{\large\textbf{IS-521 Activity Proposal}\hfill
                \textbf{Your Name}} \\
         {\phantom{} \hfill \textbf{GitHub ID}} \\
         {\phantom{} \hfill Due Date: April 15, 2017} \\
%%% Header ends

\section{Activity Overview}

Briefly describe what this activity is about, and \emph{why} you are
proposing this activity. (활동에 대한 간략한 설명과 왜 이 활동을
제안하는지에 대한 이유를 서술.)

\section{Exercises}

Describe a series of exercises that students will carry out. (학생들이 하게
될 연습문제를 순차적으로 서술.)

\begin{exercise}

  In this exercise, you do ...

\end{exercise}

\begin{exercise}

  In this exercise, you do ...

\end{exercise}

\begin{exercise}

  In this exercise, you do ...

\end{exercise}

\section{Expected Solutions}

Put expected solutions here.
(예상되는 답안에 대해서 서술.)

\bibliography{references}
\bibliographystyle{plainnat}

\end{document}
