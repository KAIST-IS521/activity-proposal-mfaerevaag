\documentclass[a4paper, 11pt]{article}

% \usepackage{kotex} % Comment this out if you are not using Hangul
\usepackage{fullpage}
\usepackage{hyperref}
\usepackage{amsthm}
\usepackage[numbers,sort&compress]{natbib}

\theoremstyle{definition}
\newtheorem{exercise}{Exercise}

\begin{document}
%%% Header starts
\noindent{\large\textbf{IS-521 Activity Proposal}\hfill
                \textbf{Markus Faerevaag}} \\
         {\phantom{} \hfill \textbf{mfaerevaag}} \\
         {\phantom{} \hfill Due Date: April 15, 2017} \\
%%% Header ends

\section{Activity Overview}

I would think it would be very interesting to have an activity where
we would inject code into a running process, on a Linux system, and
get it to execute some of our code.

\subsection{Why?}

Use cases for doing this could be reverse engineering, or development
of debuggers and reverse engineering tools. Also, I believe this gives
insight into the inner workings of some types of malware.

\subsection{Lessons?}

Students will learn how basic debuggers, using the {\tt ptrace} system
call, work, and how they and can prevent their processes from being
hooked in the same way.

\section{Exercises}

Describe a series of exercises that students will carry out. TODO

\begin{exercise}

  In this exercise, you do ...

\end{exercise}

\begin{exercise}

  In this exercise, you do ...

\end{exercise}

\begin{exercise}

  In this exercise, you do ...

\end{exercise}

\section{Expected Solutions}

I would imagine a solution similar to my referenced article, by user
called ``PicoFlamingo'' (aka. {\tt 0x00pf})~\cite{0x00sec}, which uses
around 100 lines of C code.

The author uses the {\tt ptrace} system call to hook a method in a
process, locate the process's memory and overwriting its instructions,
getting it to execute arbitrary code.

Please see reference for more information. TODO

\bibliography{references}
\bibliographystyle{plainnat}

\end{document}
