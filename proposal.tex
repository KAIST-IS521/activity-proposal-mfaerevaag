\documentclass[a4paper, 11pt]{article}

% \usepackage{kotex} % Comment this out if you are not using Hangul
\usepackage{fullpage}
\usepackage{hyperref}
\usepackage{amsthm}
\usepackage[numbers,sort&compress]{natbib}

\theoremstyle{definition}
\newtheorem{exercise}{Exercise}

\begin{document}
%%% Header starts
\noindent{\large\textbf{IS-521 Activity Proposal}\hfill
                \textbf{Markus Faerevaag}} \\
         {\phantom{} \hfill \textbf{mfaerevaag}} \\
         {\phantom{} \hfill Due Date: April 15, 2017} \\
%%% Header ends


\section{Activity Overview}
I think it could be very interesting to have an activity where we would {\bf
  inject code into a running process}, on a Linux system using {\tt\bf ptrace},
and get it to execute some of our code.

Arbitrary code execution is a common vulnerability, one specially popular by
adversaries as it easily exploited and offers a range of possibility, such as an
reverse shell.

\subsection{Why?}
Use cases for doing this could be reverse engineering, or development
of debuggers and reverse engineering tools. Also, I believe this gives
insight into the inner workings of some types of malware.

For instance, an infected program could inject code into other programs running
with the same privileges to offer remote access. There are many possibilities
for combining this activity with the vulnet/worm series.

\subsection{Lessons?}
Students will learn how basic debuggers which uses the {\tt ptrace} system call
work, and how they and can prevent their processes from being hooked in the same
way, as form of anti-debuging/anti-instrumentation.

For instance, this provides a good building block to the anti-anti-malware so
maybe the current activity could include some anti-instrumentation task, as it is
already quite short.


\section{Exercises}

I will write these exercises as an extension to the worm activity, where the
student will use it's worm implementation and add functionality where it doesn't
only propagate, but on execution it inject some code into the vulnet program,
fixing the backdoor and preventing it form being exploited again.

\begin{exercise}
  Make so the worm executes its the binary after propagation. It can be
  specified that this should only happen on the first target, to make it easier
  for the TAs to test.
\end{exercise}

\begin{exercise}
  Locate the process ID ({\tt pid}) of the running vulnet server. This could be
  done by parsing the {\tt /proc} file system, which has already been explored
  in the worm activity.
\end{exercise}

\begin{exercise}
  Using {\tt ptrace} attack to the process and hook the {\tt authenticate} method.
\end{exercise}

\begin{exercise}
  Either fix the existing backdoor, or make a new backdoor.
\end{exercise}


\section{Expected Solutions}
I would imagine a solution similar to my referenced article, by user
called ``PicoFlamingo'' (aka. {\tt 0x00pf})~\cite{0x00sec}, which uses
around 100 lines of C code.

The author uses the {\tt ptrace} system call to hook a method in a
process, locate the process's memory and overwriting its instructions,
getting it to execute arbitrary code.

In our case we would cook vulnet's {\tt authenticate} method. Vulnet's source
code could be modified to make this even easier, so the hooked function would do
as little as possible, making it easier to recreate for the worm. For instance,
the authenticate function could just receive the parameters {\tt username} and
{\tt password}, and return a boolean value.

Please see reference for more information.


\bibliography{references}
\bibliographystyle{plainnat}

\end{document}
